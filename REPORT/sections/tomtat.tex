\clearpage
\chapter*{Tóm tắt}
\addcontentsline{toc}{chapter}{Tóm tắt}

Nhận diện và phân loại phát ngôn là một trong những bài toán quan trọng của xử lý ngôn ngữ tự nhiên, nơi hệ thống cần xác định đặc trưng và sắc thái biểu đạt của các phát ngôn được tạo ra trong môi trường trực tuyến. Các phát ngôn có thể mang nhiều dạng nội dung khác nhau như tích cực, trung tính hoặc tiêu cực, và thường đa dạng và phức tạp. Điều này khiến bài toán phân loại trở nên khó xử lý, đòi hỏi mô hình phải nắm bắt được ý nghĩa, mục đích và đặc điểm ngôn ngữ ẩn sau từng câu.

Trong lĩnh vực xử lý ngôn ngữ tự nhiên, các phương pháp học máy truyền thống giữ vai trò quan trọng nhờ tính ổn định và khả năng khái quát hóa tốt trong nhiều bài toán phân loại văn bản. Các mô hình này dựa trên việc học các đặc trưng ngôn ngữ được trích xuất từ dữ liệu, sau đó sử dụng những đặc trưng đó để suy luận và phân tách các nhóm phát ngôn khác nhau. Cách tiếp cận này cho phép mô hình nắm bắt được những tín hiệu ngữ nghĩa cốt lõi trong văn bản, hỗ trợ việc nhận diện nội dung và sắc thái biểu đạt một cách nhất quán.

Trong đồ án này, ba thuật toán phân loại phổ biến gồm Support Vector Machine (SVM), Naive Bayes và Logistic Regression được triển khai nhằm khảo sát khả năng nhận biết các kiểu tính chất thường gặp trong văn bản tiếng Việt trên không gian số. Các mô hình này đại diện cho những hướng tiếp cận phổ biến trong bài toán phân loại văn bản, nhờ vào khả năng học các đặc trưng ngôn ngữ quan trọng và đưa ra dự đoán ổn định trong quá trình phân tích.
