\chapter{CƠ SỞ LÝ THUYẾT}

\section{Các phương pháp vectorize dữ liệu}
\subsection{One Hot}
% Nội dung phần khái niệm ở đây

\subsection{TF IDF}
% Nội dung phần các bước xử lý ở đây

\subsection{PhoBert}
% Nội dung phần thuật ngữ ở đây


\section{Một số phương pháp khác}
% Nội dung 


\section{Các mô hình học máy}
\subsection{Phân tích cảm xúc là gì?}
% Nội dung 1.3.1 ở đây

\subsection{Phân tích cảm xúc hoạt động thế nào?}
% Nội dung 1.3.2 ở đây


\section{Phân loại bài toán phân tích cảm xúc}
\subsection{Logistic Regression}
% Nội dung 1.4.1 ở đây

\subsection{Support Vector Machine}
 Support Vector Machine (SVM) là mô hình học máy giám sát, mục tiêu của thuật toán là tìm ra một siêu phẳng (hyperplane) sao cho có thể phân tách tối ưu các điểm dữ liệu thuộc các lớp khác nhau. “Tối ưu” ở đây nghĩa là tìm ra siêu phẳng tạo ra khoảng cách lớn nhất (margin) giữa các lớp dữ liệu. Các điểm dữ liệu có khoảng cách nhỏ nhất đến siêu mặt phẳng (các điểm gần nhất) được gọi là các vector hỗ trợ (support vectors)
Margin là độ rộng tối đa của dải không chứa bất kỳ điểm dữ liệu nào và song song với siêu phẳng.


\subsection{Phân tích cảm xúc theo cấp độ}
% Nội dung 1.4.3 ở đây

\subsection{Phân tích cảm xúc theo khía cạnh}
% Nội dung 1.4.4 ở đ
