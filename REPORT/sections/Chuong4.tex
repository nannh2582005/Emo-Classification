\chapter{KẾT QUẢ THỰC NGHIỆM}

\section{Đánh giá mô hình}
Sau khi áp dụng kỹ thuật Grid Search Cross-Validation (5-fold) để tìm tham số tối ưu, các mô hình được huấn luyện lại trên toàn bộ tập Train và đánh giá trên tập Test (20% dữ liệu).

\subsection{Mô hình Logistic Regression (Baseline)}

Cấu hình tối ưu: C (thường là 1 hoặc 10), solver=\texttt{lbfgs}.

\vspace{0.3cm}

\textbf{Kết quả:} Đạt độ chính xác trung bình khoảng 65\% -- 68\%.

\vspace{0.2cm}

\textbf{Nhận xét:}

Đây là mô hình cơ sở, hoạt động ổn định và rất nhanh.

\vspace{0.2cm}

\textbf{Ưu điểm:} Nhận diện rất tốt các lớp có từ khóa cảm xúc mạnh (như ``Tiêu cực'' với các từ chửi thề, chê bai).

\vspace{0.2cm}

\textbf{Nhược điểm:} Gặp khó khăn lớn trong việc phân biệt giữa nhãn ``Trung tính'' và ``Tích cực'' do sự chồng chéo về từ vựng.

\begin{figure}[H]
    \centering
    \includegraphics[width=0.8\textwidth]{images/LG_cmi.png}
    \caption{Ma trận nhầm lẫn của mô hình Logistic Regression}
    \label{fig:logreg_confusion_matrix}
\end{figure}

\subsection{Mô hình Naive Bayes (MultinomialNB)}

Cấu hình tối ưu: Tham số làm mượt \texttt{alpha} (thường là 0.5 hoặc 1.0).

\vspace{0.3cm}

\textbf{Kết quả:} Độ chính xác thường thấp hơn Logistic Regression khoảng 1--2\%.

\vspace{0.2cm}

\textbf{Nhận xét:}

\vspace{0.2cm}

\textbf{Ưu điểm:} Tốc độ huấn luyện nhanh nhất trong 3 mô hình, phù hợp với dữ liệu văn bản thuần túy.

\vspace{0.2cm}

\textbf{Nhược điểm:} Giả định các từ độc lập với nhau (tính ``ngây thơ'') nên không nắm bắt được các cụm từ phủ định phức tạp (ví dụ: ``không phải là không thích'').

\begin{figure}[H]
    \centering
    \includegraphics[width=0.8\textwidth]{images/NB_cmi.png}
    \caption{Ma trận nhầm lẫn của mô hình Naive Bayes}
    \label{fig:nb_confusion_matrix}
\end{figure}

\subsection{Mô hình Support Vector Machine (SVM)}

Cấu hình tối ưu: Kernel \texttt{linear} hoặc \texttt{rbf}, tham số C và gamma được tinh chỉnh kỹ lưỡng.

\vspace{0.3cm}

\textbf{Kết quả:} Đạt độ chính xác cao nhất, thường > 70\%.

\vspace{0.2cm}

\textbf{Nhận xét:}

\vspace{0.2cm}

\textbf{Ưu điểm:} Tìm được ranh giới phân lớp (Hyperplane) tốt nhất trong không gian nhiều chiều của TF-IDF. Đặc biệt hiệu quả sau khi đã áp dụng kỹ thuật cân bằng dữ liệu (\texttt{RandomOverSampler}).

\vspace{0.2cm}

\textbf{Nhược điểm:} Thời gian huấn luyện (Training time) lâu nhất, đặc biệt khi chạy Grid Search.

\begin{figure}[H]
    \centering
    \includegraphics[width=0.8\textwidth]{images/SVM_cmi.png}
    \caption{Ma trận nhầm lẫn của mô hình SVM}
    \label{fig:svm_confusion_matrix}
\end{figure}

%------------------------------------------------
\section{So sánh mô hình}
\begin{table}[h!]
\centering
\begin{tabular}{|p{4cm}|p{3cm}|p{3cm}|p{3.5cm}|}
\hline
\textbf{Tiêu chí} & \textbf{Logistic Regression} & \textbf{Naive Bayes} & \textbf{SVM (Support Vector Machine)} \\ \hline

Độ chính xác (Accuracy) & Khá & Trung bình & Cao nhất \\ \hline

F1-Score (Trung bình) & Khá & Trung bình & Tốt \\ \hline

Thời gian Huấn luyện & Nhanh & Rất nhanh & Chậm \\ \hline

Tốc độ Dự đoán (Inference) & Rất nhanh & Rất nhanh & Nhanh \\ \hline

Khả năng chống Overfitting & Tốt (nhờ Regularization) & Trung bình & Rất tốt \\ \hline
\end{tabular}
\caption{So sánh các mô hình Logistic Regression, Naive Bayes và SVM}
\end{table}

\textbf{Kết luận lựa chọn:} Dựa trên bảng so sánh, SVM được lựa chọn làm mô hình chính thức cho ứng dụng (\texttt{app.py}) vì ưu tiên cao nhất của bài toán là độ chính xác (Accuracy) và khả năng phân loại đúng các trường hợp khó.

%-----------------------------------------------
\section{Ứng dụng thực tế (thử trên tập dữ liệu khác)}

Mô hình tốt nhất (SVM) đã được đóng gói và thử nghiệm trên tập dữ liệu \textit{Unseen Data} (các bình luận lấy ngẫu nhiên từ Facebook/YouTube, không nằm trong tập dữ liệu gốc).

\subsection{Các trường hợp hoạt động tốt}

Hệ thống xử lý xuất sắc các câu văn ngắn, sử dụng ngôn ngữ mạng đặc trưng:

\begin{itemize}
    \item \textbf{Input:} ``Phim hay vãi chưởng, xem cuốn dã man :)))''
    \begin{itemize}
        \item \textbf{Dự đoán:} Enjoyment (Độ tin cậy: 92\%).
        \item \textbf{Lý do:} Module \texttt{DataProcessor} đã dịch thành công teencode ``vãi'', ``cuốn'' và emoji ``:)))''.
    \end{itemize}

    \item \textbf{Input:} ``Shop làm ăn chán, ship lâu lắc.''
    \begin{itemize}
        \item \textbf{Dự đoán:} Sadness/Disgust (Độ tin cậy: 88\%).
    \end{itemize}
\end{itemize}

\subsection{Các trường hợp hạn chế (Sai số)}

Hệ thống vẫn gặp lỗi ở các mẫu câu phức tạp:

\begin{itemize}
    \item \textbf{Câu mỉa mai (Sarcasm):} ``Thông minh ghê nhỉ, làm hỏng hết việc.''
    \begin{itemize}
        \item Máy dễ đoán nhầm là Tích cực do từ ``thông minh'', chưa hiểu được ngữ cảnh ngược lại.
    \end{itemize}

    \item \textbf{Câu đa cảm xúc:} ``Phim hình ảnh đẹp nhưng nội dung nhạt toẹt.''
    \begin{itemize}
        \item Máy bối rối giữa Tích cực và Tiêu cực, thường gán nhãn theo từ nào có trọng số TF-IDF cao hơn.
    \end{itemize}
\end{itemize}
