\chapter{Kết quả và Hướng phát triển}

\section{Kết quả}

Xây dựng thành công một hệ thống Phân loại cảm xúc tiếng Việt (Vietnamese Sentiment
Analysis) hoàn chỉnh với quy trình khép kín (End-to-End Pipeline), đạt được các tiêu chí
kỹ thuật sau:

\begin{itemize}
    \item \textbf{Kiến trúc phần mềm:} Hệ thống được thiết kế theo hướng đối tượng (OOP)
    và phân tầng (Layered Architecture).
    
    \item \textbf{Xử lý ngôn ngữ tự nhiên đặc thù:} Hệ thống giải quyết tốt các thách thức
    của tiếng Việt trên mạng xã hội thông qua bộ tiền xử lý mạnh mẽ.
    
    \item \textbf{Tối ưu hóa mô hình:} Thay vì sử dụng tham số mặc định, đồ án đã áp dụng
    kỹ thuật Grid Search Cross-Validation (5-fold) để tìm ra bộ tham số tốt nhất cho cả
    3 thuật toán: Logistic Regression, Naive Bayes và SVM.
\end{itemize}

\section{Hiệu năng trên dữ liệu}

Dựa trên kết quả thực nghiệm với bộ dữ liệu hơn 6.000 mẫu bình luận:

\begin{itemize}
    \item \textbf{Độ chính xác:} Các mô hình đạt độ chính xác trung bình từ 65\% -- 75\%.
    Trong đó, SVM (Support Vector Machine) sau khi cân bằng dữ liệu thường cho kết quả
    ổn định nhất.
    
    \item \textbf{Khả năng nhận diện:}
    \begin{itemize}
        \item Hệ thống hoạt động tốt trong việc nhận diện cảm xúc Tiêu cực (Negative)
        với độ nhạy (Recall) đạt trên 80\%. Điều này rất có ý nghĩa trong bài toán
        Social Listening (lắng nghe mạng xã hội) để cảnh báo khủng hoảng truyền thông.
        \item Nhận diện khá tốt cảm xúc Tích cực (Positive).
    \end{itemize}
    
    \item \textbf{Hạn chế:} Việc phân loại nhãn ``Trung tính'' (Neutral) vẫn còn bị
    ảnh hưởng bởi sự mơ nhạt về đặc trưng ngôn ngữ và dữ liệu mất cân bằng.
\end{itemize}

\section{Hướng phát triển tương lai}

Trong tương lai, việc mở rộng và cải thiện mô hình phân tích cảm xúc có thể bao gồm:

\begin{itemize}
    \item Sử dụng các mô hình tiên tiến hơn hoặc kết hợp nhiều mô hình để tăng độ chính xác.
    \item Thu thập và phân tích dữ liệu từ nhiều nguồn hơn để có cái nhìn toàn diện hơn
    về cảm xúc của người dùng.
    \item Áp dụng phân tích cảm xúc vào các lĩnh vực khác ngoài dịch vụ khách sạn, như
    thương mại điện tử, mạng xã hội, v.v.
\end{itemize}
