\clearpage
\chapter*{Giới thiệu}
\addcontentsline{toc}{chapter}{Giới thiệu}

Phân loại phát ngôn là một nhiệm vụ quan trọng trong phân tích nội dung người dùng trên mạng xã hội, nơi ngôn ngữ được thể hiện với nhiều sắc thái và biến thể khác nhau. Việc nhận diện mức độ và kiểu biểu đạt cảm xúc—tích cực, tiêu cực, trung tính hoặc các hình thức như mỉa mai, công kích—giúp hệ thống hiểu rõ hơn thái độ người dùng. Nhờ đó, doanh nghiệp có thể theo dõi phản hồi khách hàng, đánh giá danh tiếng thương hiệu, còn các nhà nghiên cứu có thể phân tích hành vi và xu hướng giao tiếp trực tuyến.

Kết quả đạt được trong đồ án là nền tảng để phát triển các hệ thống phân tích nâng cao hơn, phục vụ cho các bài toán như sàng lọc nội dung độc hại, dự đoán xu hướng thảo luận hay giám sát cảm xúc cộng đồng trên mạng xã hội.
\section*{Đặt vấn đề}

Sự gia tăng mạnh mẽ của lượng phát ngôn trên Internet khiến nhu cầu phân tích tự động nội dung người dùng trở nên cấp thiết. Mạng xã hội, diễn đàn và các nền tảng thảo luận trực tuyến liên tục phát sinh các ý kiến phản ánh quan điểm và cảm xúc của người dùng về nhiều chủ đề.

Phân loại phát ngôn mang lại giá trị thiết thực:

\textbf{Đối với người dùng}: Hỗ trợ nắm bắt nhanh xu hướng thảo luận và đánh giá tổng quan cảm xúc cộng đồng.

\textbf{Đối với tổ chức, doanh nghiệp}: Giúp theo dõi danh tiếng thương hiệu, phát hiện chủ đề nhạy cảm và tối ưu chiến lược truyền thông.

Tuy nhiên, bài toán gặp nhiều thách thức như:

\textbf{Khối lượng dữ liệu lớn}: không thể xử lý thủ công.

\textbf{Ngôn ngữ phức tạp}: xuất hiện teencode, emoji, từ viết tắt và các biểu đạt cảm xúc không chuẩn hóa.

Do đó, việc xây dựng hệ thống phân loại phát ngôn tự động là cần thiết nhằm hỗ trợ nhận diện nhanh chóng nội dung, theo dõi xu hướng và trích xuất thông tin hữu ích từ lượng dữ liệu khổng lồ trên mạng xã hội.

\section*{Lý do chọn đề tài}

Phát ngôn trực tuyến ngày càng đa dạng và ảnh hưởng sâu rộng đến cộng đồng cũng như hoạt động của doanh nghiệp. Tuy nhiên, sự bùng nổ dữ liệu cùng đặc điểm ngôn ngữ không chuẩn hóa khiến việc phân tích thủ công trở nên không khả thi. Nội dung tiêu cực có thể lan nhanh, tác động đến danh tiếng thương hiệu và gây khó khăn cho công tác quản lý cộng đồng.

Chính vì vậy, việc nghiên cứu và xây dựng hệ thống phân loại phát ngôn tiếng Việt mang ý nghĩa thực tiễn và khoa học. Hệ thống này hỗ trợ doanh nghiệp hiểu khách hàng, giúp nhà quản lý giám sát nội dung và là công cụ phục vụ nghiên cứu hành vi ngôn ngữ. Từ nhu cầu thực tế đó, nhóm quyết định chọn đề tài “Nhận diện và phân loại phát ngôn trên mạng xã hội”.
\section*{Phát biểu bài toán}

Mục tiêu của nghiên cứu là xây dựng mô hình có khả năng nhận diện và phân loại các phát ngôn trên mạng xã hội tiếng Việt theo nhóm nội dung hoặc sắc thái cảm xúc, tập trung vào ba lớp chính: tích cực, tiêu cực và trung tính.

Dữ liệu sử dụng gồm gần 7.000 phát ngôn tiếng Việt đã được gán nhãn cảm xúc. Để giải quyết bài toán, đồ án triển khai và đánh giá ba mô hình học máy truyền thống: \textit{Support Vector Machine (SVM)}, \textit{Naive Bayes} và \textit{Logistic Regression}.
\section*{Cấu trúc đồ án}

\noindent\textbf{PHẦN 1: CƠ SỞ LÝ THUYẾT}

Trình bày các khái niệm liên quan đến bài toán xử lý ngôn ngữ tự nhiên, những đặc điểm của ngôn ngữ trên mạng xã hội, cùng các phương pháp phân loại văn bản cơ bản thường được sử dụng trong NLP.

\noindent\textbf{PHẦN 2: THU THẬP, XỬ LÝ VÀ KHÁM PHÁ DỮ LIỆU}

Mô tả quy trình thu thập dữ liệu phát ngôn từ mạng xã hội, các bước tiền xử lý văn bản như làm sạch, chuẩn hóa, xử lý emoji, teencode và stopwords, đồng thời thực hiện phân tích khám phá dữ liệu (EDA) nhằm hiểu rõ các đặc trưng của tập dữ liệu.

\noindent\textbf{PHẦN 3: XÂY DỰNG MÔ HÌNH PHÂN TÍCH}

Trình bày các mô hình học máy được sử dụng trong nghiên cứu gồm Support Vector Machine (SVM), Naive Bayes và Logistic Regression; mô tả cách trích xuất đặc trưng, thiết lập mô hình và quy trình huấn luyện.

\noindent\textbf{PHẦN 4: KẾT QUẢ THỰC NGHIỆM}

Tiến hành thử nghiệm, trình bày và so sánh kết quả mô hình dựa trên các thước đo đánh giá, trực quan hóa kết quả và đưa ra nhận xét về hiệu quả phân loại phát ngôn.
