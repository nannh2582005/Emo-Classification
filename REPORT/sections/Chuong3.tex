\chapter{XÂY DỰNG MÔ HÌNH PHÂN TÍCH}

\section{Module Loader}
\subsection{Các thư viện chính}

Hệ thống sử dụng các thư viện mã nguồn mở sau:

\begin{itemize}
    \item \textbf{Scikit-learn (sklearn):} 
    Thư viện cốt lõi dùng để xây dựng các mô hình học máy như 
    \textit{Logistic Regression}, \textit{Naive Bayes} và \textit{SVM}. 
    Thư viện cung cấp đầy đủ công cụ để:
    \begin{itemize}
        \item trích xuất đặc trưng văn bản bằng \textit{TF--IDF Vectorizer};
        \item tiền xử lý dữ liệu (train--test split, chuẩn hóa);
        \item huấn luyện và đánh giá mô hình.
    \end{itemize}

    Các tham số quan trọng thường dùng:
    \begin{itemize}
        \item \texttt{C} (SVM): điều chỉnh mức phạt khi mô hình phân loại sai;
        \item \texttt{kernel} (SVM): lựa chọn hàm nhân (\textit{linear}, \textit{rbf}, \dots);
        \item \texttt{alpha} (Naive Bayes): hệ số làm trơn Laplace;
        \item \texttt{max\_features} (TF--IDF): giới hạn số lượng đặc trưng.
    \end{itemize}

    \item \textbf{Pandas:} 
    Dùng để đọc dữ liệu từ các định dạng như \texttt{.csv} và \texttt{.xlsx},
    đồng thời hỗ trợ thao tác linh hoạt trên \textit{DataFrame} 
    (lọc, gộp, thống kê, biến đổi dữ liệu).

    Tham số quan trọng:
    \begin{itemize}
        \item \texttt{read\_csv()}, \texttt{read\_excel()}: đọc dữ liệu;
        \item \texttt{dropna()}: loại bỏ hàng bị thiếu;
        \item \texttt{astype()}: chuyển kiểu dữ liệu.
    \end{itemize}

    \item \textbf{PyVi:}
    Thư viện xử lý tiếng Việt chuyên dụng, dùng để tách từ (\textit{tokenization}),
    giúp văn bản tiếng Việt được phân chia theo cụm từ có nghĩa trước khi đưa vào mô hình. 
    Điều này cải thiện độ chính xác của TF--IDF và mô hình phân loại.

    Phương thức chính:
    \begin{itemize}
        \item \texttt{ViTokenizer.tokenize(sentence)}: tách từ trong câu đầu vào.
    \end{itemize}

    \item \textbf{Joblib:}
    Dùng để tuần tự hóa (serialize) mô hình đã huấn luyện và lưu xuống đĩa,
    cho phép tái sử dụng mô hình mà không cần huấn luyện lại.

    Hàm quan trọng:
    \begin{itemize}
        \item \texttt{joblib.dump(model, path)}: lưu mô hình;
        \item \texttt{joblib.load(path)}: tải mô hình.
    \end{itemize}

    \item \textbf{Matplotlib \& Seaborn:}
    Hai thư viện trực quan hóa dữ liệu giúp vẽ các biểu đồ thống kê 
    và ma trận nhầm lẫn (\textit{Confusion Matrix}).  
    Trong đồ án, nhóm sử dụng chúng để đánh giá trực quan hiệu năng mô hình.

    Các hàm phổ biến:
    \begin{itemize}
        \item \texttt{plt.figure(figsize)}: tạo khung hình;
        \item \texttt{sns.heatmap()}: vẽ ma trận nhầm lẫn;
        \item \texttt{plt.xlabel()}, \texttt{plt.ylabel()}: thêm nhãn trục.
    \end{itemize}

    \item \textbf{OpenPyXL:}
    Thư viện hỗ trợ đọc và ghi dữ liệu vào định dạng Excel (\texttt{.xlsx}),
    được sử dụng khi bộ dữ liệu ban đầu ở dạng Excel hoặc khi cần xuất dữ liệu đã xử lý.

    Hàm quan trọng:
    \begin{itemize}
        \item \texttt{load\_workbook()}: đọc file Excel;
        \item \texttt{worksheet.append()}: thêm dòng dữ liệu;
        \item \texttt{workbook.save()}: lưu tệp Excel.
    \end{itemize}

\end{itemize}


\section{Module Preprocessor}
% Nội dung 3.2

\section{Module feature}
% Nội dung 3.3

\section{Tối ưu tham số}
\subsection{Logistic Regression}
% Nội dung 3.4.1

\subsection{Support Vector Machine}
% Nội dung 3.4.2

\subsection{Naive Bayes}
% Nội dung 3.4.3

\section{Huấn luyện mô hình}
\subsection{Logistic Regression}
% Nội dung 3.5.1

\subsection{Support Vector Machine}
% Nội dung 3.5.2

\subsection{Naive Bayes}
% Nội dung 3.5.3